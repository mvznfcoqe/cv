%-------------------------
% Resume in Latex
% Author : Dmitri Moskalenko
% Based off of: https://github.com/sb2nov/resume
% License : MIT
%------------------------


\documentclass[letterpaper,11pt]{article}

\usepackage{latexsym}
\usepackage[empty]{fullpage}
\usepackage{titlesec}
\usepackage{marvosym}
\usepackage[usenames,dvipsnames]{color}
\usepackage{verbatim}
\usepackage{enumitem}
\usepackage[hidelinks]{hyperref}
\usepackage{fancyhdr}
\usepackage[russian]{babel}
\usepackage{tabularx}
\input{glyphtounicode}


%----------FONT OPTIONS----------
% sans-serif
% \usepackage[sfdefault]{FiraSans}
% \usepackage[sfdefault]{roboto}
% \usepackage[sfdefault]{noto-sans}
% \usepackage[default]{sourcesanspro}

% serif
% \usepackage{CormorantGaramond}
% \usepackage{charter}


\pagestyle{fancy}
\fancyhf{} % clear all header and footer fields
\fancyfoot{}
\renewcommand{\headrulewidth}{0pt}
\renewcommand{\footrulewidth}{0pt}

% Adjust margins
\addtolength{\oddsidemargin}{-0.5in}
\addtolength{\evensidemargin}{-0.5in}
\addtolength{\textwidth}{1in}
\addtolength{\topmargin}{-.5in}
\addtolength{\textheight}{1.0in}

\urlstyle{same}

\raggedbottom
\raggedright
\setlength{\tabcolsep}{0in}

% Sections formatting
\titleformat{\section}{
  \vspace{-4pt}\scshape\raggedright\large
}{}{0em}{}[\color{black}\titlerule \vspace{-5pt}]

% Ensure that generate pdf is machine readable/ATS parsable
\pdfgentounicode=1

%-------------------------
% Custom commands
\newcommand{\resumeItem}[1]{
  \item\small{
    {#1 \vspace{-2pt}}
  }
}

\newcommand{\resumeSubheading}[4]{
  \vspace{-2pt}\item
    \begin{tabular*}{0.97\textwidth}[t]{l@{\extracolsep{\fill}}r}
      \textbf{#1} & #2 \\
      \textit{\small#3} & \textit{\small #4} \\
    \end{tabular*}\vspace{-7pt}
}

\newcommand{\resumeSubSubheading}[2]{
    \item
    \begin{tabular*}{0.97\textwidth}{l@{\extracolsep{\fill}}r}
      \textit{\small#1} & \textit{\small #2} \\
    \end{tabular*}\vspace{-7pt}
}

\newcommand{\resumeProjectHeading}[2]{
    \item
    \begin{tabular*}{0.97\textwidth}{l@{\extracolsep{\fill}}r}
      \small#1 & #2 \\
    \end{tabular*}\vspace{-7pt}
}

\newcommand{\resumeSubItem}[1]{\resumeItem{#1}\vspace{-4pt}}

\renewcommand\labelitemii{$\vcenter{\hbox{\tiny$\bullet$}}$}

\newcommand{\resumeSubHeadingListStart}{\begin{itemize}[leftmargin=0.15in, label={}]}
\newcommand{\resumeSubHeadingListEnd}{\end{itemize}}
\newcommand{\resumeItemListStart}{\begin{itemize}}
\newcommand{\resumeItemListEnd}{\end{itemize}\vspace{-5pt}}

%-------------------------------------------
%%%%%%  RESUME STARTS HERE  %%%%%%%%%%%%%%%%%%%%%%%%%%%%


\begin{document}

%----------HEADING----------
% \begin{tabular*}{\textwidth}{l@{\extracolsep{\fill}}r}
%   \textbf{\href{http://sourabhbajaj.com/}{\Large Sourabh Bajaj}} & Email : \href{mailto:sourabh@sourabhbajaj.com}{sourabh@sourabhbajaj.com}\\
%   \href{http://sourabhbajaj.com/}{http://www.sourabhbajaj.com} & Mobile : +1-123-456-7890 \\
% \end{tabular*}

\begin{center}
  \textbf{\Huge \scshape Дмитрий Москаленко} \\ \vspace{1pt}
  \href{mailto:dmitrii.moskalenko2@gmail.com}{\underline{dmitrii.moskalenko2@gmail.com}} $|$
  \href{https://www.linkedin.com/in/dmitri-moskalenko/}{\underline{linkedin.com/in/dmitri-moskalenko}} $|$
  \href{https://github.com/mvznfcoqe}{\underline{github.com/mvznfcoqe}}
\end{center}


\section{О себе}
Frontend-разработчик с 4+ годами коммерческого опыта в экосистемах Vue/React, работал в командах от 2 до 18 человек. Выстраивал полный цикл разработки: от сбора требований до автоматизированных релизов с высоким покрытием тестами, CI/CD и постановке процессов в команде. Специализируюсь на модернизации legacy-систем, масштабируемой архитектуре и платформенной разработке.

%-----------EXPERIENCE-----------
\section{Опыт работы}
\resumeSubHeadingListStart

% -----------Multiple Positions Heading-----------
% \resumeSubSubheading
% {Software Engineer I}{Oct 2014 - Sep 2016}
% \resumeItemListStart
% \resumeItem{Apache Beam}
% {Apache Beam is a unified model for defining both batch and streaming data-parallel processing pipelines}
% \resumeItemListEnd
% \resumeSubHeadingListEnd
% -------------------------------------------

\resumeSubheading
{Frontend Developer}{Декабрь 2024 -- Сейчас}
{Rocket Science}{Удаленно}
\resumeItemListStart
\resumeItem{Разработал AI-интеграции для edtech-платформы: инструменты генерации контента, проверки задач учеников для авторов курсов и контекстный AI-помощник для учеников, ускорив создание образовательных материалов}
\resumeItem{Внедрил E2E тестирование на Playwright, увеличив количество автотестов с 0 до 30+ для покрытия критических пользовательских сценариев, что сократило количество регрессионных багов и позволило безопасно производить обновления проекта}
\resumeItem{Провел миграцию с Vue CLI на Vite, сократив время запуска локальной рабочей среды на 90\% (с 52 сек до 5 сек) и время сборки на 58\% (с 194 сек до 80 сек) и улучшив опыт разработки для 5+ инженеров }
\resumeItem{Спроектировал и внедрил модульную архитектуру с динамическими импортами, уменьшив initial bundle на 70\% (с 2.8 MB до 850 KB) и Time to Interactive на 55\%, что обеспечило быструю загрузку для пользователей с медленным интернетом}
\resumeItemListEnd

\resumeSubheading
{Frontend Developer}{Ноябрь 2023 -- Декабрь 2024}
{Seenday}{Удаленно}
\resumeItemListStart
\resumeItem{Модернизировал платёжный флоу для сервиса по продаже фотографий (1.5K фотографов, 70K DAU в сезон) с несколькими сценариями покупки (альбомы, фотосессии, наборы) и гибкой системой подарков, что привело к увеличению среднего чека на 20\% для фотографов}
\resumeItem{Провёл миграцию фотомаркетплейса (70K+ DAU, 120K LOC) с PHP-монолита на Vue, ускорив time-to-market для сложных клиентских задач на 60\% и сократив онбординг с двух недель до 5–7 дней}
\resumeItem{Внедрил правило написания unit-тестов (Vitest) для нового функционала, достигнув 60\% покрытия и повысив надёжность написанного кода}
\resumeItem{Менторил 3 trainee-разработчиков, проводя code review и еженедельные 1-on-1, что помогло им вырасти в junior разработчиков внутри компании за 3 месяца }
\resumeItemListEnd

\resumeSubheading
{Frontend Developer}{Сентябрь 2021 -- Октябрь 2023}
{AFSMD}{Удаленно}
\resumeItemListStart
\resumeItem{Разработал функционал учёта рабочего времени с автоматическим расчётом стоимости, сократив потерю оплачиваемых часов на 12\%}
\resumeItem{Повысил качество кодовой базы, подключив Storybook для 50+ компонентов и расширив использование TypeScript с 28\% до 85\%, что упростило рефакторинг и поиск ошибок на этапе сборки проекта}
\resumeItem{Внедрил contract-first подход к API (OpenAPI, Orval, MSW) с генерацией типизированного клиента и автоматическим созданием моков, устранив проблемы синхронизации между командами и обеспечив параллельную разработку}
\resumeItemListEnd

\resumeSubHeadingListEnd


% %-----------PROJECTS-----------
% \section{Projects}
% \resumeSubHeadingListStart
% \resumeProjectHeading
% {\textbf{Gitlytics} $|$ \emph{Python, Flask, React, PostgreSQL, Docker}}{June 2020 -- Present}
% \resumeItemListStart
% \resumeItem{Developed a full-stack web application using with Flask serving a REST API with React as the frontend}
% \resumeItem{Implemented GitHub OAuth to get data from user’s repositories}
% \resumeItem{Visualized GitHub data to show collaboration}
% \resumeItem{Used Celery and Redis for asynchronous tasks}
% \resumeItemListEnd
% \resumeProjectHeading
% {\textbf{Simple Paintball} $|$ \emph{Spigot API, Java, Maven, TravisCI, Git}}{May 2018 -- May 2020}
% \resumeItemListStart
% \resumeItem{Developed a Minecraft server plugin to entertain kids during free time for a previous job}
% \resumeItem{Published plugin to websites gaining 2K+ downloads and an average 4.5/5-star review}
% \resumeItem{Implemented continuous delivery using TravisCI to build the plugin upon new a release}
% \resumeItem{Collaborated with Minecraft server administrators to suggest features and get feedback about the plugin}
% \resumeItemListEnd
% \resumeSubHeadingListEnd


%-----------EDUCATION-----------
\section{Образование}
\resumeSubHeadingListStart
\resumeSubheading
{Воронежский Государственный Технический Университет}{Россия, Воронеж}
{Информационные системы и технологии}{2021 -- 2025}
\resumeSubHeadingListEnd


%
%-----------PROGRAMMING SKILLS-----------
\section{Технические Навыки}
\begin{itemize}[leftmargin=0.15in, label={}]
  \small{\item{
                \textbf{Languages}{: JavaScript, TypeScript, HTML, CSS } \\
                \textbf{Runtimes}{: Node.js, Bun, Browser } \\
                \textbf{Frameworks}{: Vue, React, Hono, NestJS } \\
                \textbf{DevOps}{: Git, Gitlab/Github CI, Docker, Ansible, Prometheus, Grafana, Caddy/Nginx, Proxmox} \\
                \textbf{Libraries}{: GSAP, Tanstack Query/Virtual/Table, Hono, Pinia, Effector, Redux, Zustand, Radix/Reka/Headless UI, Tailwind CSS, UnoCSS, ApexCharts, Chart.js, Orval, MSW, Grammy, Storybook, ofetch, Vitest, Playwright}
          }}
\end{itemize}


%-------------------------------------------
\end{document}
